\documentclass[12pt]{report}

\usepackage{amssymb,amsmath}
\usepackage{fullpage}
\usepackage{listings}      % for showing code nicely
\usepackage{courier}       % monospace font for listings

\pagestyle{empty}

\def\Z{{\mathbb Z}}
\def\Q{{\mathbb Q}}
\def\C{{\mathbb C}}
\def\calO{{\mathcal O}}    % avoid redefining \O
\def\R{{\mathbb R}}
\def\N{{\mathbb N}}

% settings for Python listings
\lstset{
  basicstyle=\small\ttfamily,
  language=Python,
  numbers=left,
  numberstyle=\tiny,
  frame=single,
  breaklines=true,
  showstringspaces=false
}

\begin{document}

\large

\begin{center}
Math 440/540 Homework 2
\end{center}

\normalsize

\begin{enumerate}

\item (\# 1.7 in Stein)
\begin{enumerate}
\item (using python or SageMath)  Compute $\pi(x)=\#\{p\leq x \mid p \text{ prime}\}$ and $x/\log(x)$ for $x\in \{1000, 10000, 100000\}$.

\bigskip

\begin{lstlisting}
# Example Python (SageMath) code
print(f'pi(1000)={prime_pi(1000)}')
print(f'1000/log(1000)={1000 / log(1000).n()}')
print(f'pi(10000)={prime_pi(10000)}')
print(f'10000/log(10000)={10000 / log(10000).n()}')
print(f'pi(100000)={prime_pi(100000)}')
print(f'1000000/log(1000000)={1000000 / log(1000000).n()}')
\end{lstlisting}

\smallskip

(Example output:)
\begin{verbatim}
pi(1000)=168
1000/log(1000)=144.764827301084
pi(10000)=1229
10000/log(10000)=1085.73620475813
pi(100000)=9592
1000000/log(1000000)=72382.4136505420
\end{verbatim}

\item (by hand) The Prime Number Theorem states that $\pi(x)$ is asymptotic to $x/\log(x)$.  What does that mean (in your own words)?

\paragraph{Answer:} That means that as $x\to\infty$ the ratio $\dfrac{\pi(x)}{x/\log x}\to 1$. This means that as we get to larger
numbers, $\pi(x)$ can be better approximated by $x/log(x)$ This also means that as x approaches infinity, the distance between primes generally
becomes greater and greater.
\end{enumerate}

\item(\#2.7 in Stein)
(by hand) Find complete sets of residues modulo $7$ satisfying each condition below.

\begin{enumerate}
\item all elements are nonnegative
\[\{0, 1, 2, 3, 4, 5, 6 \}\]
\item all elements are odd
\[\{1, 3, 5, 7, 9, 11, 13\}\]
\item all elements are even
\[\{2, 4, 6, 8, 10, 12, 14, \}\]
\item all elements are prime
\[\{2, 3, 5, 11, 7, 13, 29\}\]
\item all elements are multiples of $3$
\[\{0, 3, 6, 9, 12, 15, 18\}\] since $gcd(7,3)=1$
\end{enumerate}

\item 
(by hand)
\begin{enumerate}
\item Show how the Extended Euclidean Algorithm can be used to compute the inverse of $3$ modulo $20$.  
\paragraph{Answer:} First we can expand $gcd(20,3)$
\[20=3*6+2 => 3=2*1+1 => 2=1*2+0\]
Then we have
\[1=3-2*1\] and \[ 2=20-3*6\]
so we can subsitute and get
\[1=3-(20-3*6)*1=21-1*20=7*3-1*20\]
thus
\[7*3 \equiv 1(mod20)\]
so the inverse of 3 mod 20 is 7
\item Use your result from part (a) to solve the linear equation $3x\equiv 4 \pmod{20}$.
\paragraph{Answer:}We will multiply both sides by the inverse 7
\[7*3x\equiv7*4(mod20)=>x\equiv28\equiv8(mod20)\]
which can be checked
\[3*8=24\equiv4(mod20)\]
\end{enumerate}

\item (*) (by hand) A troop of $17$ monkeys store their bananas in $11$ piles of equal size, each containing more than $1$ banana, with a twelfth pile of $6$ left over.  When they divide the bananas into $17$ piles of equal size, none remain.  


\begin{enumerate}
\item How can you set up this problem to use the Chinese Remainder Theorem?
\item \paragraph{Answer:} To start we can set this up as $N=11k+6$ with N being
total number of bananas and $k>1$ being the size of piles. When divided into
equal piles we have $N\equiv0(mod17)$. We can write the original set up as
$N\equiv6(mod11)$ which sets us up to use the chinese remainder theorem.

\item What is the smallest number of bananas they can have?
\paragraph{Answer:} to start we can have $N=17t$ for the second situation which is
the total amount of bananas. From there we can set $17t\equiv6(mod11)$
Looking quickly we can see that $17\equiv6(mod11)$ so $6t\equiv6(mod11)$
and we can further simplify to $t\equiv1(mod11)$. Therefore $t=1+11m$ and
$N=17(1+11m)=17+187m$. If we try $m=0$ we get that k=1 which we said before
isn't true, so the next lowest is $m=1$ which gives us
\[N=17+187=204\] So the smallest valid number of bananas is 204
\end{enumerate}

\item (using python or SageMath) Euler's Theorem states that $x^{\varphi(n)}\equiv 1 \pmod n$ when $\gcd(x,n)=1$.  Consider what happens when $\gcd(x,n)\neq1$. For $n=100$, determine all possible values of $x^{\varphi(n)}$ modulo $n$ for $x$ satisfying $\gcd(x,n)\neq1$.  Note any patterns you observe.
\bigskip
\begin{lstlisting}
# Example Python (SageMath) code
n = 100
p_n = euler_phi(n)
l1 = {}
for x in range(0, n):
    if gcd(n,x) != 1:
        l1[x] = power_mod(x, p_n, n)
        
print(l1)
\end{lstlisting}
\smallskip
(Example output:)
\begin{lstlisting}[breaklines=true, breakatwhitespace=false]
{0: 0, 2: 76, 4: 76, 5: 25, 6: 76, 8: 76, 10: 0, 12: 76, 14: 76, 15: 25, 16: 76, 18: 76, 20: 0, 22: 76, 24: 76, 25: 25, 26: 76, 28: 76, 30: 0, 32: 76, 34: 76, 35: 25, 36: 76, 38: 76, 40: 0, 42: 76, 44: 76, 45: 25, 46: 76, 48: 76, 50: 0, 52: 76, 54: 76, 55: 25, 56: 76, 58: 76, 60: 0, 62: 76, 64: 76, 65: 25, 66: 76, 68: 76, 70: 0, 72: 76, 74: 76, 75: 25, 76: 76, 78: 76, 80: 0, 82: 76, 84: 76, 85: 25, 86: 76, 88: 76, 90: 0, 92: 76, 94: 76, 95: 25, 96: 76, 98: 76}
\end{lstlisting}
Which we can see only has values of 0, 25, and 76. Also they seem to follow the pattern, 0, 76, 76, 25, 76, 76, 0.

\end{enumerate}

\end{document}
