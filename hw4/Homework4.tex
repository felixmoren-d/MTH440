\documentclass[12pt]{report}

\usepackage{amssymb, fullpage, amsmath}

\pagestyle{empty}

\def\Z{{\mathbb Z}}
\def\Q{{\mathbb Q}}
\def\C{{\mathbb C}}
\def\O{{\mathcal O}}
\def\R{{\mathbb R}}
\def\N{{\mathbb N}}

\begin{document}

\large

\begin{center}
Math 440/540 Homework 4
\end{center}

\normalsize

\begin{enumerate}

\item
(by hand)
Let $n=55$.  Determine the set of allowable $e$ such that $(n,e)$ can be used as the public key in an RSA cryptosystem.
\paragraph{Answer:}
First we need to find $\phi(55)$ which we can do by saying
\[\phi(55)=\phi(5)*\phi(11)=(5-1)(11-1)=40\]
Now we need to find all numbers who fulfill $gcd(40,n)=1$ where $2\leq n < 40$
This gives us
\[[3,7,9,11,13,17,19,21,23,27,29,31,33,37,39]\]
\item
(using SageMath) Nikita underestimated you as an adversary, and decided to use small numbers for her RSA cryptosystem out of laziness...  In particular, she published the following public key:
\[
(4460543,1407259).
\]

\begin{enumerate}
\item
Determine Nikita's secret decryption key $d$.
\item 
You intercept the following encoded message to Nikita, $$(507330, 766099, 704785, 3564829).$$
Decode the secret message!
\begin{verbatim}
n = 4460543
e = 1407259

factored = factor(n)
p,q = [f[0]for f in factored]
print(p,q)

phi = (p-1) * (q-1)

print(phi)

d = inverse_mod(e, phi)

print(d)

cmessage = [507330, 766099, 704785, 3564829]

decode = [pow(c, d, n) for c in cmessage]
print(decode)

alphabet = " ABCDEFGHIJKLMNOPQRSTUVWXYZ"

text = []
for t in decode:
    word = []
    while t > 0:
        t, r = divmod(t, 27)
        word.append(alphabet[r])
    if len(word)<4:
        word.append(' ')
    text.append(word)
decoded = ''
for w in text:
    for c in w:
        decoded += c
print(decoded)



--------

2111 2113
4456320
19
[171900, 14094, 3881, 407648]
RUSH IS THE BEST

\end{verbatim}
\end{enumerate}

\item (\#4.1 in Stein) 
(by hand) Calculate the following Legendre symbols:
\[
\left(\frac{3}{97}\right), \left(\frac{3}{389}\right), \left(\frac{22}{11}\right), \left(\frac{5!}{7}\right).
\]
\paragraph{Answer}
With quadratic reciprocity we can determine that for an odd prime p
\[
\left(\frac{3}{p}\right)=\left(\frac{p}{3}\right)(-1)^{\frac{p-1}{2}}
\]
Using that we can determine the solutions

\[
\left(\frac{3}{97}\right)=\left(\frac{97}{3}\right)
\]
Since we have an odd prime
\[
\left(\frac{97}{3}\right)=>97(mod3)\equiv1(mod3)
\]
And since $1^2(mod3)\equiv1(mod3)$
\[\left(\frac{3}{97}\right)=1\]

\item (*) (\#4.3 in Stein)
(by hand) Use Gauss's Quadratic Reciprocity Law (Thm. 4.1.7 in book) to prove that for $p\geq 5$ prime, 
\[
\left(\frac{3}{p}\right) = \begin{cases} 1& \mbox{ if } p \equiv 1,11 \pmod{12} \\ -1& \mbox{ if } p \equiv 5,7 \pmod{12}. \end{cases}
\]
\paragraph{Answer}
First we have quadratic reciprocity which says for odd primes p,
\[
\left(\frac{3}{p}\right)=\left(\frac{p}{3}\right)(-1)^{\frac{3-1}{2}*\frac{p-1}{2}}=\left(\frac{p}{3}\right)(-1)^{\frac{p-1}{2}}
\]
And we know that $\left(\frac{p}{3}\right)=1$ if $p\equiv1(mod3)$ and -1 if $p\equiv2(mod3)$
Then we can look at $(-1)^{\frac{p-1}{2}}$ and determine that it equals 1 if $p\equiv1(mod4)$
and equals -1 if $p\equiv3(mod4)$. From there we can use CRT we can combine. For the cases that equal 1 when
$p\equiv1(mod3)$ and $p\equiv1(mod4)$ we get $p\equiv1(mod12)$ then we have $p\equiv2(mod3)$ 
and $p\equiv3(mod4)$ which gives us $p\equiv11(mod12)$. Then for the cases that equal -1 we have
$p\equiv1(mod3)$ and $p\equiv3(mod4)$ which gives us $p\equiv5(mod12)$ and $p\equiv2(mod3)$ and $p\equiv1(mod4)$
which gives us $p\equiv7(mod12)$. Therefor we have proved the above statement is true.

\item (*) (\#4.7 in Stein) 
(by hand with calculator) Use Gauss's Quadratic Reciprocity Law (Thm. 4.1.7 in book) to determine the number of positive integers $x<2^{13}$ that satisfy the equation
\[
x^2 \equiv 5 \pmod{2^{13}-1}.
\]
Note: you can use the fact that the number $2^{13}-1$ is prime.

\paragraph{Answer}
We can start with the fact that $2^{13}-1=8191$ which is prime. Then we can use quadratic reciprocity
to get from $\left(\frac{5}{8191}\right)$ to $\left(\frac{8191}{5}\right)$. From there we can get to
$8191\equiv1(mod5)$ and due to the properties of the Legendre symbol we know that that means we have
exactly 2 distinct solutions.

\end{enumerate}

\end{document}
