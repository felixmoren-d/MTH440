\documentclass[12pt]{report}

\usepackage{amssymb, fullpage, amsmath}

\pagestyle{empty}

\def\Z{{\mathbb Z}}
\def\Q{{\mathbb Q}}
\def\C{{\mathbb C}}
\def\O{{\mathcal O}}
\def\R{{\mathbb R}}
\def\N{{\mathbb N}}

\begin{document}

\large

\begin{center}
Math 440/540 Homework 3
\end{center}

\normalsize

\begin{enumerate}

\item(\#2.30 in Stein)
(by hand) Use fast exponentiation together with Euler's Theorem to compute the last two digits of $3^{45}$.
\paragraph{Answer:} First we set up the problem we are trying to solve in modulo terms which is
\[3^{45} (mod 100)\]
From there we can use Eulers Theorem
\[3^{\phi(100)}\equiv1(mod100)=>3^{40}\equiv1(mod100)\]
Using that we can then reduce our equation down to 
\[3^{5}(mod100)\] since $3^{40}\equiv1(mod100)$
Then we can use fast exponentiation to calculate our final value
\[3^2=9 => 3^4=91 => 3^5=243\equiv 43 mod(100)\]

\item(\#2.32 in Stein)
(using python or SageMath) Find the proportion of primes $p<1000$ such that $2$ is a primitive root modulo $p$ (i.e. $2$ has order $p-1$ modulo $p$).

\begin{verbatim}
primes = prime_range(1,1000)
prim_count = sum(primitive_root(p)==2 for p in primes)
print(f"num of primes {len(primes)}")
print(f"primitive root count {prim_count}")
print(f"proportion of primes {prim_count/len(primes)}")
======
num of primes 168
primitive root count 67
proportion of primes 0.39880952380952384
\end{verbatim}


\item (using python or SageMath)
Determine the smallest composite $n$ that is a base $2$ pseudoprime (i.e. $2^{n-1} \equiv 1 \pmod{n}$).  What is the least base $a>1$ for which the Pseudoprimality Theorem proves $n$ is composite?
\begin{verbatim}
n = 2
while True:
    if not is_prime(n):
        
        if mod(2^(n-1),n)==1:
            break
    n+=1
print(n)
===
341
\end{verbatim}
\begin{verbatim}
n = 341
for a in range(2,341):
    if mod(a^(n-1),n)!=1:
        print(a)
        break
===
3
\end{verbatim}
\item(\#3.4 in Stein) (by hand with calculator or using python or SageMath)
You and Nikita wish to agree on a secret key $s$ using the Diffie-Hellman key exchange.  Nikita announces that $p=3793$ and $g=7$.  She secretly chooses a number $n<p$ and tells you that $g^n\equiv 454 \pmod{p}$.  You choose the random number $m=1208$.  What is the secret key?
\begin{verbatim}
p = 3793
g = 7
m=1208
key = power_mod(454,m,p)
print(key)
===
2156
\end{verbatim}
\item (using python or SageMath)
Let $n=14380057$.  
\begin{enumerate}
\item
Determine the largest block size you can use to encode strings of text for use with RSA with this $n$.  
\item
Using the block size in (a), encode the phrase `HAPPY HALLOWEEN'  as a list of numbers using our method from class.
\item If Nikita announces her public RSA key to be $(14380057,7)$, encrypt your answer from (b) using Nikita's public key.
\end{enumerate}
\begin{verbatim}
n = 14380057
max_block = floor(log(n,10)/2)
print(max_block)
msg = "HAPPY HALLOWEEN"
word_list = [msg[i:i+3] for i in range(0, len(msg), 3)]
    
print(word_list)

def encode_character(c):
    if c == " ":
        return 0
    else:
        return ord(c)-64
encode_list = []
for block in word_list:
    num = 0
    i = 0
    for c in block:
        num += encode_character(c)*27^i
        i += 1
    encode_list.append(num)
print(encode_list)
e = 7
encrypt_list = [pow(b, e, n) for b in encode_list] 
print(encrypt_list)
===
3
['HAP', 'PY ', 'HAL', 'LOW', 'EEN']
[11699, 691, 8783, 17184, 10346]
[6961752, 2354377, 9032872, 3150324, 8876056]
\end{verbatim}
\end{enumerate}

\end{document}